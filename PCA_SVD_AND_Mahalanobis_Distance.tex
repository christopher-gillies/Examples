\documentclass[]{article}
\usepackage{lmodern}
\usepackage{amssymb,amsmath}
\usepackage{ifxetex,ifluatex}
\usepackage{fixltx2e} % provides \textsubscript
\ifnum 0\ifxetex 1\fi\ifluatex 1\fi=0 % if pdftex
  \usepackage[T1]{fontenc}
  \usepackage[utf8]{inputenc}
\else % if luatex or xelatex
  \ifxetex
    \usepackage{mathspec}
  \else
    \usepackage{fontspec}
  \fi
  \defaultfontfeatures{Ligatures=TeX,Scale=MatchLowercase}
\fi
% use upquote if available, for straight quotes in verbatim environments
\IfFileExists{upquote.sty}{\usepackage{upquote}}{}
% use microtype if available
\IfFileExists{microtype.sty}{%
\usepackage{microtype}
\UseMicrotypeSet[protrusion]{basicmath} % disable protrusion for tt fonts
}{}
\usepackage[margin=1in]{geometry}
\usepackage{hyperref}
\hypersetup{unicode=true,
            pdftitle={PCA, SVD and Mahalanobis distance},
            pdfauthor={Christopher Gillies},
            pdfborder={0 0 0},
            breaklinks=true}
\urlstyle{same}  % don't use monospace font for urls
\usepackage{color}
\usepackage{fancyvrb}
\newcommand{\VerbBar}{|}
\newcommand{\VERB}{\Verb[commandchars=\\\{\}]}
\DefineVerbatimEnvironment{Highlighting}{Verbatim}{commandchars=\\\{\}}
% Add ',fontsize=\small' for more characters per line
\usepackage{framed}
\definecolor{shadecolor}{RGB}{248,248,248}
\newenvironment{Shaded}{\begin{snugshade}}{\end{snugshade}}
\newcommand{\KeywordTok}[1]{\textcolor[rgb]{0.13,0.29,0.53}{\textbf{{#1}}}}
\newcommand{\DataTypeTok}[1]{\textcolor[rgb]{0.13,0.29,0.53}{{#1}}}
\newcommand{\DecValTok}[1]{\textcolor[rgb]{0.00,0.00,0.81}{{#1}}}
\newcommand{\BaseNTok}[1]{\textcolor[rgb]{0.00,0.00,0.81}{{#1}}}
\newcommand{\FloatTok}[1]{\textcolor[rgb]{0.00,0.00,0.81}{{#1}}}
\newcommand{\ConstantTok}[1]{\textcolor[rgb]{0.00,0.00,0.00}{{#1}}}
\newcommand{\CharTok}[1]{\textcolor[rgb]{0.31,0.60,0.02}{{#1}}}
\newcommand{\SpecialCharTok}[1]{\textcolor[rgb]{0.00,0.00,0.00}{{#1}}}
\newcommand{\StringTok}[1]{\textcolor[rgb]{0.31,0.60,0.02}{{#1}}}
\newcommand{\VerbatimStringTok}[1]{\textcolor[rgb]{0.31,0.60,0.02}{{#1}}}
\newcommand{\SpecialStringTok}[1]{\textcolor[rgb]{0.31,0.60,0.02}{{#1}}}
\newcommand{\ImportTok}[1]{{#1}}
\newcommand{\CommentTok}[1]{\textcolor[rgb]{0.56,0.35,0.01}{\textit{{#1}}}}
\newcommand{\DocumentationTok}[1]{\textcolor[rgb]{0.56,0.35,0.01}{\textbf{\textit{{#1}}}}}
\newcommand{\AnnotationTok}[1]{\textcolor[rgb]{0.56,0.35,0.01}{\textbf{\textit{{#1}}}}}
\newcommand{\CommentVarTok}[1]{\textcolor[rgb]{0.56,0.35,0.01}{\textbf{\textit{{#1}}}}}
\newcommand{\OtherTok}[1]{\textcolor[rgb]{0.56,0.35,0.01}{{#1}}}
\newcommand{\FunctionTok}[1]{\textcolor[rgb]{0.00,0.00,0.00}{{#1}}}
\newcommand{\VariableTok}[1]{\textcolor[rgb]{0.00,0.00,0.00}{{#1}}}
\newcommand{\ControlFlowTok}[1]{\textcolor[rgb]{0.13,0.29,0.53}{\textbf{{#1}}}}
\newcommand{\OperatorTok}[1]{\textcolor[rgb]{0.81,0.36,0.00}{\textbf{{#1}}}}
\newcommand{\BuiltInTok}[1]{{#1}}
\newcommand{\ExtensionTok}[1]{{#1}}
\newcommand{\PreprocessorTok}[1]{\textcolor[rgb]{0.56,0.35,0.01}{\textit{{#1}}}}
\newcommand{\AttributeTok}[1]{\textcolor[rgb]{0.77,0.63,0.00}{{#1}}}
\newcommand{\RegionMarkerTok}[1]{{#1}}
\newcommand{\InformationTok}[1]{\textcolor[rgb]{0.56,0.35,0.01}{\textbf{\textit{{#1}}}}}
\newcommand{\WarningTok}[1]{\textcolor[rgb]{0.56,0.35,0.01}{\textbf{\textit{{#1}}}}}
\newcommand{\AlertTok}[1]{\textcolor[rgb]{0.94,0.16,0.16}{{#1}}}
\newcommand{\ErrorTok}[1]{\textcolor[rgb]{0.64,0.00,0.00}{\textbf{{#1}}}}
\newcommand{\NormalTok}[1]{{#1}}
\usepackage{graphicx,grffile}
\makeatletter
\def\maxwidth{\ifdim\Gin@nat@width>\linewidth\linewidth\else\Gin@nat@width\fi}
\def\maxheight{\ifdim\Gin@nat@height>\textheight\textheight\else\Gin@nat@height\fi}
\makeatother
% Scale images if necessary, so that they will not overflow the page
% margins by default, and it is still possible to overwrite the defaults
% using explicit options in \includegraphics[width, height, ...]{}
\setkeys{Gin}{width=\maxwidth,height=\maxheight,keepaspectratio}
\IfFileExists{parskip.sty}{%
\usepackage{parskip}
}{% else
\setlength{\parindent}{0pt}
\setlength{\parskip}{6pt plus 2pt minus 1pt}
}
\setlength{\emergencystretch}{3em}  % prevent overfull lines
\providecommand{\tightlist}{%
  \setlength{\itemsep}{0pt}\setlength{\parskip}{0pt}}
\setcounter{secnumdepth}{0}
% Redefines (sub)paragraphs to behave more like sections
\ifx\paragraph\undefined\else
\let\oldparagraph\paragraph
\renewcommand{\paragraph}[1]{\oldparagraph{#1}\mbox{}}
\fi
\ifx\subparagraph\undefined\else
\let\oldsubparagraph\subparagraph
\renewcommand{\subparagraph}[1]{\oldsubparagraph{#1}\mbox{}}
\fi

%%% Use protect on footnotes to avoid problems with footnotes in titles
\let\rmarkdownfootnote\footnote%
\def\footnote{\protect\rmarkdownfootnote}

%%% Change title format to be more compact
\usepackage{titling}

% Create subtitle command for use in maketitle
\newcommand{\subtitle}[1]{
  \posttitle{
    \begin{center}\large#1\end{center}
    }
}

\setlength{\droptitle}{-2em}
  \title{PCA, SVD and Mahalanobis distance}
  \pretitle{\vspace{\droptitle}\centering\huge}
  \posttitle{\par}
  \author{Christopher Gillies}
  \preauthor{\centering\large\emph}
  \postauthor{\par}
  \predate{\centering\large\emph}
  \postdate{\par}
  \date{11/16/2017}


\begin{document}
\maketitle

\section{Multivariate normal
distribution}\label{multivariate-normal-distribution}

\begin{equation}
f(x)=\frac{1}{\sqrt{(2\pi)^n|\boldsymbol\Sigma|}}
\exp\left(-\frac{1}{2}({x}-{\mu})^T{\boldsymbol\Sigma}^{-1}({x}-{\mu})
\right)
\end{equation}

If we assume the data are zero-centered then:

\begin{equation}
f(x)=\frac{1}{\sqrt{(2\pi)^n|\boldsymbol\Sigma|}}
\exp\left(-\frac{1}{2}({x})^T{\boldsymbol\Sigma}^{-1}({x})
\right)
\end{equation}

\section{\texorpdfstring{where \(x\) is an \(m\)-dimensional
vector.}{where x is an m-dimensional vector.}}\label{where-x-is-an-m-dimensional-vector.}

\subsection{Generate a random sample}\label{generate-a-random-sample}

\begin{Shaded}
\begin{Highlighting}[]
\NormalTok{S =}\StringTok{ }\KeywordTok{matrix}\NormalTok{(}\KeywordTok{c}\NormalTok{(}\DecValTok{1}\NormalTok{,}\FloatTok{0.75}\NormalTok{,}\FloatTok{0.75}\NormalTok{,}\DecValTok{2}\NormalTok{),}\DataTypeTok{ncol=}\DecValTok{2}\NormalTok{)}
\NormalTok{x =}\StringTok{ }\KeywordTok{mvrnorm}\NormalTok{(}\DataTypeTok{n =} \DecValTok{1000}\NormalTok{, }\DataTypeTok{mu=}\KeywordTok{c}\NormalTok{(}\DecValTok{0}\NormalTok{,}\DecValTok{0}\NormalTok{), }\DataTypeTok{Sigma=}\NormalTok{S)}

\KeywordTok{ggplot}\NormalTok{(}\KeywordTok{data.frame}\NormalTok{()) +}\StringTok{ }\KeywordTok{geom_point}\NormalTok{(}\KeywordTok{aes}\NormalTok{(}\DataTypeTok{x=}\NormalTok{x[,}\DecValTok{1}\NormalTok{],}\DataTypeTok{y=}\NormalTok{x[,}\DecValTok{2}\NormalTok{])) +}\StringTok{ }\KeywordTok{scale_x_continuous}\NormalTok{(}\DataTypeTok{limits=}\KeywordTok{c}\NormalTok{(-}\DecValTok{5}\NormalTok{,}\DecValTok{5}\NormalTok{))  +}\StringTok{ }\KeywordTok{scale_y_continuous}\NormalTok{(}\DataTypeTok{limits=}\KeywordTok{c}\NormalTok{(-}\DecValTok{5}\NormalTok{,}\DecValTok{5}\NormalTok{)) +}
\StringTok{  }\KeywordTok{ggtitle}\NormalTok{(}\StringTok{"Random sample from a  bivariate normal distribution"}\NormalTok{) +}\StringTok{ }\KeywordTok{xlab}\NormalTok{(}\StringTok{"x1"}\NormalTok{) +}\StringTok{ }\KeywordTok{ylab}\NormalTok{(}\StringTok{"x2"}\NormalTok{)}
\end{Highlighting}
\end{Shaded}

\begin{verbatim}
## Warning: Removed 2 rows containing missing values (geom_point).
\end{verbatim}

\includegraphics{PCA_SVD_AND_Mahalanobis_Distance_files/figure-latex/unnamed-chunk-1-1.pdf}

\subsection{Perform PCA}\label{perform-pca}

In principal component analysis, we compute the covariance of the
features we are studying, and then compute the eigenvectors of this
matrix. Let a matrix \(X \in \mathbb{R}^{n \times m}\) be a matrix,
where we have \(n\) observations for \(m\) features.

\begin{equation}
X_{c} = X - {\mu_j}
\end{equation}

\(X_c\) is the centered matrix of \(X\), where we subtract the mean
(\(\mu_j\)) of each column (feature) from the data and
\(j \in \{1,..m\}\).

\begin{Shaded}
\begin{Highlighting}[]
\NormalTok{S.sample =}\StringTok{ }\KeywordTok{cov}\NormalTok{(x)}

\NormalTok{e.decomp =}\StringTok{ }\KeywordTok{eigen}\NormalTok{(S.sample)}

\NormalTok{e.decomp$vectors %*%}\StringTok{ }\KeywordTok{diag}\NormalTok{(e.decomp$values) %*%}\StringTok{ }\KeywordTok{t}\NormalTok{(e.decomp$vectors)}
\end{Highlighting}
\end{Shaded}

\begin{verbatim}
##           [,1]      [,2]
## [1,] 1.0189460 0.7819877
## [2,] 0.7819877 2.0131314
\end{verbatim}

\begin{Shaded}
\begin{Highlighting}[]
\NormalTok{x.proj =}\StringTok{ }\NormalTok{x %*%}\StringTok{ }\NormalTok{e.decomp$vectors}
\KeywordTok{ggplot}\NormalTok{(}\KeywordTok{data.frame}\NormalTok{()) +}\StringTok{ }\KeywordTok{geom_point}\NormalTok{(}\KeywordTok{aes}\NormalTok{(}\DataTypeTok{x=}\NormalTok{x.proj[,}\DecValTok{1}\NormalTok{],}\DataTypeTok{y=}\NormalTok{x.proj[,}\DecValTok{2}\NormalTok{])) +}\StringTok{ }\KeywordTok{scale_x_continuous}\NormalTok{(}\DataTypeTok{limits=}\KeywordTok{c}\NormalTok{(-}\DecValTok{5}\NormalTok{,}\DecValTok{5}\NormalTok{))  +}\StringTok{ }\KeywordTok{scale_y_continuous}\NormalTok{(}\DataTypeTok{limits=}\KeywordTok{c}\NormalTok{(-}\DecValTok{5}\NormalTok{,}\DecValTok{5}\NormalTok{))}
\end{Highlighting}
\end{Shaded}

\begin{verbatim}
## Warning: Removed 5 rows containing missing values (geom_point).
\end{verbatim}

\includegraphics{PCA_SVD_AND_Mahalanobis_Distance_files/figure-latex/unnamed-chunk-2-1.pdf}

\begin{Shaded}
\begin{Highlighting}[]
\NormalTok{e.decomp$values}
\end{Highlighting}
\end{Shaded}

\begin{verbatim}
## [1] 2.4426486 0.5894288
\end{verbatim}

\begin{Shaded}
\begin{Highlighting}[]
\KeywordTok{var}\NormalTok{(x.proj[,}\DecValTok{1}\NormalTok{])}
\end{Highlighting}
\end{Shaded}

\begin{verbatim}
## [1] 2.442649
\end{verbatim}

\begin{Shaded}
\begin{Highlighting}[]
\KeywordTok{var}\NormalTok{(x.proj[,}\DecValTok{2}\NormalTok{])}
\end{Highlighting}
\end{Shaded}

\begin{verbatim}
## [1] 0.5894288
\end{verbatim}

Notice that the variance of the projected data matches the eigenvalues
of the covariance matrix of \(X\). The projection of x onto its
principal components simply rotates the data.

\begin{Shaded}
\begin{Highlighting}[]
\NormalTok{svd.sample.cov =}\StringTok{ }\KeywordTok{svd}\NormalTok{(S.sample)}


\NormalTok{x.center =}\StringTok{ }\KeywordTok{scale}\NormalTok{(x,}\DataTypeTok{scale =} \OtherTok{FALSE}\NormalTok{)}

\NormalTok{svd.x =}\StringTok{ }\KeywordTok{svd}\NormalTok{(x.center)}


\NormalTok{x.proj}\FloatTok{.2} \NormalTok{=}\StringTok{ }\NormalTok{x.center %*%}\StringTok{ }\NormalTok{svd.x$v}
\NormalTok{x.proj}\FloatTok{.3} \NormalTok{=}\StringTok{ }\NormalTok{x.center %*%}\StringTok{ }\NormalTok{svd.sample.cov$v}

\KeywordTok{plot}\NormalTok{(x.proj}\FloatTok{.2}\NormalTok{[,}\DecValTok{1}\NormalTok{],x.proj}\FloatTok{.3}\NormalTok{[,}\DecValTok{1}\NormalTok{])}
\end{Highlighting}
\end{Shaded}

\includegraphics{PCA_SVD_AND_Mahalanobis_Distance_files/figure-latex/unnamed-chunk-3-1.pdf}

\begin{Shaded}
\begin{Highlighting}[]
\KeywordTok{plot}\NormalTok{(x.proj}\FloatTok{.2}\NormalTok{[,}\DecValTok{2}\NormalTok{],x.proj}\FloatTok{.3}\NormalTok{[,}\DecValTok{2}\NormalTok{])}
\end{Highlighting}
\end{Shaded}

\includegraphics{PCA_SVD_AND_Mahalanobis_Distance_files/figure-latex/unnamed-chunk-3-2.pdf}

\begin{Shaded}
\begin{Highlighting}[]
\NormalTok{svd.x$d^}\DecValTok{2} \NormalTok{/}\StringTok{ }\NormalTok{(}\DecValTok{1000} \NormalTok{-}\StringTok{ }\DecValTok{1}\NormalTok{)}
\end{Highlighting}
\end{Shaded}

\begin{verbatim}
## [1] 2.4426486 0.5894288
\end{verbatim}

\begin{Shaded}
\begin{Highlighting}[]
\NormalTok{svd.sample.cov$d}
\end{Highlighting}
\end{Shaded}

\begin{verbatim}
## [1] 2.4426486 0.5894288
\end{verbatim}

The same eigenvectors and eigenvalues are computed from the matrix
x.center and the covariance matrix. The singlar values \(\sigma_i\) of x
and the eigenvalues \(\lambda_i\) of COV\((X)\) are related as follows:

\begin{equation}
\lambda_i = \frac{\sigma_i^2}{n-1} 
\end{equation}

\begin{equation}
\text{COV} \left [ X \right ] = \frac{1}{n-1}X^TX 
\end{equation}

\begin{equation}
X = U \Sigma V^T 
\end{equation}

\begin{equation}
X^TX = (U \Sigma V^T)^T (U \Sigma V^T)
\end{equation}

\begin{equation}
X^TX = V \Sigma U^TU \Sigma V^T
\end{equation}

\begin{equation}
X^TX = V \Sigma \Sigma V^T = V \Sigma^2 V^T
\end{equation}

\begin{equation}
\frac{1}{n-1}X^TX = \frac{1}{n-1} V \Sigma^2 V^T \rightarrow \frac{1}{n-1} \Sigma^2 = \Lambda
\end{equation}

where \(\Lambda\) is the diagonal matrix of eigenvalues of
\(\text{COV} \left [ X \right ]\). Also note that \(V\) is a unitary
matrix.

This is the same formula as above for the relationship between the
singular values of X and the eigenvalues of its covariance matrix.

\subsection{What happens if we scale X before running
PCA?}\label{what-happens-if-we-scale-x-before-running-pca}

\begin{Shaded}
\begin{Highlighting}[]
\NormalTok{x.scaled =}\StringTok{ }\KeywordTok{scale}\NormalTok{(x)}

\NormalTok{cov.scaled.x =}\StringTok{ }\KeywordTok{cov}\NormalTok{(x.scaled)}
\NormalTok{cov.scaled.x}
\end{Highlighting}
\end{Shaded}

\begin{verbatim}
##           [,1]      [,2]
## [1,] 1.0000000 0.5459945
## [2,] 0.5459945 1.0000000
\end{verbatim}

\begin{Shaded}
\begin{Highlighting}[]
\NormalTok{S.sample}
\end{Highlighting}
\end{Shaded}

\begin{verbatim}
##           [,1]      [,2]
## [1,] 1.0189460 0.7819877
## [2,] 0.7819877 2.0131314
\end{verbatim}

\begin{Shaded}
\begin{Highlighting}[]
\KeywordTok{ggplot}\NormalTok{(}\KeywordTok{data.frame}\NormalTok{()) +}\StringTok{ }\KeywordTok{geom_point}\NormalTok{(}\KeywordTok{aes}\NormalTok{(}\DataTypeTok{x=}\NormalTok{x[,}\DecValTok{1}\NormalTok{],}\DataTypeTok{y=}\NormalTok{x[,}\DecValTok{2}\NormalTok{])) +}\StringTok{ }\KeywordTok{scale_x_continuous}\NormalTok{(}\DataTypeTok{limits=}\KeywordTok{c}\NormalTok{(-}\DecValTok{5}\NormalTok{,}\DecValTok{5}\NormalTok{))  +}\StringTok{ }\KeywordTok{scale_y_continuous}\NormalTok{(}\DataTypeTok{limits=}\KeywordTok{c}\NormalTok{(-}\DecValTok{5}\NormalTok{,}\DecValTok{5}\NormalTok{)) +}\StringTok{ }\KeywordTok{ggtitle}\NormalTok{(}\StringTok{"Before scaling"}\NormalTok{)}
\end{Highlighting}
\end{Shaded}

\begin{verbatim}
## Warning: Removed 2 rows containing missing values (geom_point).
\end{verbatim}

\includegraphics{PCA_SVD_AND_Mahalanobis_Distance_files/figure-latex/unnamed-chunk-4-1.pdf}

\begin{Shaded}
\begin{Highlighting}[]
\KeywordTok{ggplot}\NormalTok{(}\KeywordTok{data.frame}\NormalTok{()) +}\StringTok{ }\KeywordTok{geom_point}\NormalTok{(}\KeywordTok{aes}\NormalTok{(}\DataTypeTok{x=}\NormalTok{x.scaled[,}\DecValTok{1}\NormalTok{],}\DataTypeTok{y=}\NormalTok{x.scaled[,}\DecValTok{2}\NormalTok{])) +}\StringTok{ }\KeywordTok{scale_x_continuous}\NormalTok{(}\DataTypeTok{limits=}\KeywordTok{c}\NormalTok{(-}\DecValTok{5}\NormalTok{,}\DecValTok{5}\NormalTok{))  +}\StringTok{ }\KeywordTok{scale_y_continuous}\NormalTok{(}\DataTypeTok{limits=}\KeywordTok{c}\NormalTok{(-}\DecValTok{5}\NormalTok{,}\DecValTok{5}\NormalTok{)) +}\StringTok{ }\KeywordTok{ggtitle}\NormalTok{(}\StringTok{"After scaling"}\NormalTok{)}
\end{Highlighting}
\end{Shaded}

\includegraphics{PCA_SVD_AND_Mahalanobis_Distance_files/figure-latex/unnamed-chunk-4-2.pdf}

\begin{Shaded}
\begin{Highlighting}[]
\NormalTok{scaled.svd =}\StringTok{ }\KeywordTok{svd}\NormalTok{(cov.scaled.x)}

\NormalTok{x.scaled.proj =}\StringTok{ }\NormalTok{x.scaled %*%}\StringTok{ }\NormalTok{scaled.svd$v }
\KeywordTok{ggplot}\NormalTok{(}\KeywordTok{data.frame}\NormalTok{()) +}\StringTok{ }\KeywordTok{geom_point}\NormalTok{(}\KeywordTok{aes}\NormalTok{(}\DataTypeTok{x=}\NormalTok{x.scaled.proj[,}\DecValTok{1}\NormalTok{],}\DataTypeTok{y=}\NormalTok{x.scaled.proj[,}\DecValTok{2}\NormalTok{])) +}\StringTok{ }\KeywordTok{scale_x_continuous}\NormalTok{(}\DataTypeTok{limits=}\KeywordTok{c}\NormalTok{(-}\DecValTok{5}\NormalTok{,}\DecValTok{5}\NormalTok{))  +}\StringTok{ }\KeywordTok{scale_y_continuous}\NormalTok{(}\DataTypeTok{limits=}\KeywordTok{c}\NormalTok{(-}\DecValTok{5}\NormalTok{,}\DecValTok{5}\NormalTok{)) +}\StringTok{ }\KeywordTok{ggtitle}\NormalTok{(}\StringTok{"After projection scaled x"}\NormalTok{) }
\end{Highlighting}
\end{Shaded}

\includegraphics{PCA_SVD_AND_Mahalanobis_Distance_files/figure-latex/unnamed-chunk-5-1.pdf}

\begin{Shaded}
\begin{Highlighting}[]
\NormalTok{svd.sample.cov$d}
\end{Highlighting}
\end{Shaded}

\begin{verbatim}
## [1] 2.4426486 0.5894288
\end{verbatim}

\begin{Shaded}
\begin{Highlighting}[]
\NormalTok{scaled.svd$d}
\end{Highlighting}
\end{Shaded}

\begin{verbatim}
## [1] 1.5459945 0.4540055
\end{verbatim}

It is interesting to note that the angle between the principal axes
changes.

\begin{equation}
cos \theta = \frac{v \dot w}{\lVert v \rVert \lVert x \rVert}
\end{equation}

\subsection{Data reconstruction}\label{data-reconstruction}

\begin{Shaded}
\begin{Highlighting}[]
\NormalTok{x.projection =}\StringTok{ }\NormalTok{svd.sample.cov$v %*%}\StringTok{ }\KeywordTok{t}\NormalTok{(x.center)}
\NormalTok{x.reconstruction =}\StringTok{ }\KeywordTok{t}\NormalTok{(svd.sample.cov$v %*%}\StringTok{ }\NormalTok{x.projection)}
\NormalTok{x.reconstruction.partial =}\StringTok{ }\KeywordTok{t}\NormalTok{(svd.sample.cov$v %*%}\StringTok{ }\KeywordTok{rbind}\NormalTok{(x.projection[}\DecValTok{1}\NormalTok{,], }\DecValTok{0}\NormalTok{))}

\KeywordTok{cor}\NormalTok{(x.reconstruction[,}\DecValTok{1}\NormalTok{],x.center[,}\DecValTok{1}\NormalTok{])}
\end{Highlighting}
\end{Shaded}

\begin{verbatim}
## [1] 1
\end{verbatim}

\begin{Shaded}
\begin{Highlighting}[]
\KeywordTok{cor}\NormalTok{(x.reconstruction[,}\DecValTok{2}\NormalTok{],x.center[,}\DecValTok{2}\NormalTok{])}
\end{Highlighting}
\end{Shaded}

\begin{verbatim}
## [1] 1
\end{verbatim}

\begin{Shaded}
\begin{Highlighting}[]
\KeywordTok{cor}\NormalTok{(x.reconstruction[,}\DecValTok{1}\NormalTok{],x.reconstruction.partial[,}\DecValTok{1}\NormalTok{])}
\end{Highlighting}
\end{Shaded}

\begin{verbatim}
## [1] 0.7453868
\end{verbatim}

\begin{Shaded}
\begin{Highlighting}[]
\KeywordTok{cor}\NormalTok{(x.reconstruction[,}\DecValTok{2}\NormalTok{],x.reconstruction.partial[,}\DecValTok{2}\NormalTok{])}
\end{Highlighting}
\end{Shaded}

\begin{verbatim}
## [1] 0.9654741
\end{verbatim}

Notince that the reconstruction works perfectly, and the partial
reconstruction works fairly well.

\subsection{Whitening transform}\label{whitening-transform}

\begin{Shaded}
\begin{Highlighting}[]
\NormalTok{x.proj.white =}\StringTok{ }\NormalTok{x.center %*%}\StringTok{ }\NormalTok{svd.sample.cov$v %*%}\StringTok{ }\KeywordTok{solve}\NormalTok{(}\KeywordTok{diag}\NormalTok{(}\KeywordTok{sqrt}\NormalTok{(svd.sample.cov$d)))}
\KeywordTok{var}\NormalTok{(x.proj.white[,}\DecValTok{1}\NormalTok{])}
\end{Highlighting}
\end{Shaded}

\begin{verbatim}
## [1] 1
\end{verbatim}

\begin{Shaded}
\begin{Highlighting}[]
\KeywordTok{var}\NormalTok{(x.proj.white[,}\DecValTok{2}\NormalTok{])}
\end{Highlighting}
\end{Shaded}

\begin{verbatim}
## [1] 1
\end{verbatim}

\begin{Shaded}
\begin{Highlighting}[]
\KeywordTok{ggplot}\NormalTok{(}\KeywordTok{data.frame}\NormalTok{()) +}\StringTok{ }\KeywordTok{geom_point}\NormalTok{(}\KeywordTok{aes}\NormalTok{(}\DataTypeTok{x=}\NormalTok{x.proj.white[,}\DecValTok{1}\NormalTok{],}\DataTypeTok{y=}\NormalTok{x.proj.white[,}\DecValTok{2}\NormalTok{])) +}\StringTok{ }\KeywordTok{scale_x_continuous}\NormalTok{(}\DataTypeTok{limits=}\KeywordTok{c}\NormalTok{(-}\DecValTok{4}\NormalTok{,}\DecValTok{4}\NormalTok{))  +}\StringTok{ }\KeywordTok{scale_y_continuous}\NormalTok{(}\DataTypeTok{limits=}\KeywordTok{c}\NormalTok{(-}\DecValTok{4}\NormalTok{,}\DecValTok{4}\NormalTok{)) +}\StringTok{ }\KeywordTok{ggtitle}\NormalTok{(}\StringTok{"Whitening Transfrom"}\NormalTok{)}
\end{Highlighting}
\end{Shaded}

\includegraphics{PCA_SVD_AND_Mahalanobis_Distance_files/figure-latex/unnamed-chunk-8-1.pdf}

\begin{Shaded}
\begin{Highlighting}[]
\NormalTok{w_sqrt =}\StringTok{ }\NormalTok{svd.sample.cov$v %*%}\StringTok{ }\KeywordTok{diag}\NormalTok{(}\KeywordTok{sqrt}\NormalTok{(svd.sample.cov$d)) %*%}\StringTok{ }\KeywordTok{t}\NormalTok{(svd.sample.cov$v)}
\NormalTok{w_inv_sqrt =}\StringTok{ }\KeywordTok{solve}\NormalTok{(w_sqrt)}
\NormalTok{w_inv_sqrt}\FloatTok{.2} \NormalTok{=}\StringTok{ }\KeywordTok{t}\NormalTok{(svd.sample.cov$v) %*%}\StringTok{ }\KeywordTok{solve}\NormalTok{(}\KeywordTok{diag}\NormalTok{(}\KeywordTok{sqrt}\NormalTok{(svd.sample.cov$d))) %*%}\StringTok{ }\NormalTok{svd.sample.cov$v}
\NormalTok{x.proj.white}\FloatTok{.2}\NormalTok{=}\StringTok{ }\NormalTok{x.center %*%}\StringTok{ }\NormalTok{w_inv_sqrt}\FloatTok{.2}
\KeywordTok{ggplot}\NormalTok{(}\KeywordTok{data.frame}\NormalTok{()) +}\StringTok{ }\KeywordTok{geom_point}\NormalTok{(}\KeywordTok{aes}\NormalTok{(}\DataTypeTok{x=}\NormalTok{x.proj.white}\FloatTok{.2}\NormalTok{[,}\DecValTok{1}\NormalTok{],}\DataTypeTok{y=}\NormalTok{x.proj.white}\FloatTok{.2}\NormalTok{[,}\DecValTok{2}\NormalTok{])) +}\StringTok{ }\KeywordTok{scale_x_continuous}\NormalTok{(}\DataTypeTok{limits=}\KeywordTok{c}\NormalTok{(-}\DecValTok{4}\NormalTok{,}\DecValTok{4}\NormalTok{))  +}\StringTok{ }\KeywordTok{scale_y_continuous}\NormalTok{(}\DataTypeTok{limits=}\KeywordTok{c}\NormalTok{(-}\DecValTok{4}\NormalTok{,}\DecValTok{4}\NormalTok{)) +}\StringTok{ }\KeywordTok{ggtitle}\NormalTok{(}\StringTok{"Whitening Transfrom 2"}\NormalTok{)}
\end{Highlighting}
\end{Shaded}

\begin{verbatim}
## Warning: Removed 1 rows containing missing values (geom_point).
\end{verbatim}

\includegraphics{PCA_SVD_AND_Mahalanobis_Distance_files/figure-latex/unnamed-chunk-8-2.pdf}

\begin{Shaded}
\begin{Highlighting}[]
\KeywordTok{cov}\NormalTok{(x.proj.white}\FloatTok{.2}\NormalTok{)}
\end{Highlighting}
\end{Shaded}

\begin{verbatim}
##               [,1]          [,2]
## [1,]  1.000000e+00 -7.409104e-17
## [2,] -7.409104e-17  1.000000e+00
\end{verbatim}

\begin{Shaded}
\begin{Highlighting}[]
\KeywordTok{cov}\NormalTok{(x.proj.white)}
\end{Highlighting}
\end{Shaded}

\begin{verbatim}
##              [,1]         [,2]
## [1,] 1.000000e+00 2.528297e-17
## [2,] 2.528297e-17 1.000000e+00
\end{verbatim}

The above are two different whitening transforms. Please note that:

\begin{equation}
\text{COV}[X] = V \Lambda V^T
\end{equation}

In a diagonal matrix \(\Lambda\) we can take the square roots by just
taking the square root of the diagonal elements

\begin{equation}
\text{COV}[X] = (V \Lambda^{1/2} V^T)^2 = V \Lambda^{1/2} (V^T V) \Lambda^{1/2} V^T = V \Lambda V^T
\end{equation}

Since, \(V^T V = I\). That is \(V^T = V^{-1}\).

\begin{equation}
\text{COV}[X]^{1/2} = V \Lambda^{1/2} V^T 
\end{equation}

Now \((ABC)^-1 = C^{-1}B^{-1}A^{-1}\). Therefore

\begin{equation}
\text{COV}[X]^{-1/2} = (V \Lambda^{1/2} V^T )^{-1} = V^{-1} \Lambda^{-1/2} (V^{-1})^-1 = V^{T} \Lambda^{-1/2} V
\end{equation}

Since \(V^T = V^{-1}\).

\section{\texorpdfstring{Mahalanobis
distance\(^2\)}{Mahalanobis distance\^{}2}}\label{mahalanobis-distance2}

The Mahalanobis of a centered vector from the origin is:

\begin{equation}
D_{M}(x)^2 = x^T S^{-1} x
\end{equation}

where \(S\) is the covariance matrix of the vector \(x\).

\begin{Shaded}
\begin{Highlighting}[]
\KeywordTok{t}\NormalTok{(x.center[}\DecValTok{1}\NormalTok{,]) %*%}\StringTok{ }\KeywordTok{solve}\NormalTok{(S.sample) %*%}\StringTok{ }\NormalTok{x.center[}\DecValTok{1}\NormalTok{,]}
\end{Highlighting}
\end{Shaded}

\begin{verbatim}
##           [,1]
## [1,] 0.8426972
\end{verbatim}

\begin{Shaded}
\begin{Highlighting}[]
\KeywordTok{sum}\NormalTok{(x.proj.white[}\DecValTok{1}\NormalTok{,]^}\DecValTok{2}\NormalTok{)}
\end{Highlighting}
\end{Shaded}

\begin{verbatim}
## [1] 0.8426972
\end{verbatim}

\begin{Shaded}
\begin{Highlighting}[]
\KeywordTok{sum}\NormalTok{(x.proj.white}\FloatTok{.2}\NormalTok{[}\DecValTok{1}\NormalTok{,]^}\DecValTok{2}\NormalTok{)}
\end{Highlighting}
\end{Shaded}

\begin{verbatim}
## [1] 0.8426972
\end{verbatim}

\begin{Shaded}
\begin{Highlighting}[]
\NormalTok{x}\FloatTok{.1}\NormalTok{.proj.scaled =}\StringTok{ }\NormalTok{x.center[}\DecValTok{1}\NormalTok{,] %*%}\StringTok{ }\NormalTok{svd.sample.cov$v %*%}\StringTok{ }\KeywordTok{solve}\NormalTok{(}\KeywordTok{diag}\NormalTok{(}\KeywordTok{sqrt}\NormalTok{(svd.sample.cov$d)))}
\KeywordTok{sum}\NormalTok{(x}\FloatTok{.1}\NormalTok{.proj.scaled^}\DecValTok{2}\NormalTok{)}
\end{Highlighting}
\end{Shaded}

\begin{verbatim}
## [1] 0.8426972
\end{verbatim}

Mahalanobis distance\(^2\) is the equal to the \(\lVert x \rVert^2\) in
the Whitened space.

The norm of a vector

\begin{equation}
\lVert x \rVert^2 = \sum{x_i^2} = x^Tx
\end{equation}\begin{equation}
(x^T)^T = x
\end{equation}

\begin{equation}
D_{M}(x)^2 = x^T \text{COV}[X]^{-1} x
\end{equation}

\begin{equation}
\text{COV}[X]^{-1} = (V \Lambda V^T )^{-1} = V^T \Lambda^{-1} V
\end{equation}

The Mahalanobis distance\(^2\) can be written as

\begin{equation}
D_{M}(x)^2 = x^T V^T \Lambda^{-1} V x =  x^T V^T \Lambda^{-1/2}  \Lambda^{-1/2} V x = (x^T V^T \Lambda^{-1/2})  (\Lambda^{-1/2} V x) = (\Lambda^{-1/2} V x)^T (\Lambda^{-1/2} V x)
\end{equation}\begin{equation}
= (\Lambda^{-1/2} V x)^T (\Lambda^{-1/2} V x) = \lVert \Lambda^{-1/2} V x \rVert^2 = \lVert x^T V^T \Lambda^{-1/2} \rVert^2 = D_{M}(x)^2 
\end{equation}

The final equality \(\lVert x^T V^T \Lambda^{-1/2} \rVert^2\) shows that
the \(D_{M}(x)^2\) is the same as projecting the \(x\) onto its
principal components, then scaling each axis by the square root of its
eigenvalue (if the eigenvalue is the variance then the sqrt(eigenvalue)
is like the standard deviation) and finally taking the norm of the
scaled projected x. A eigenvalue is the variance of the data projected
onto its corresponding eigenvector, so to scale it you divide by the
standard deviation.

\section{What about PCA and SVD in the case where there are more
variables than
observations?}\label{what-about-pca-and-svd-in-the-case-where-there-are-more-variables-than-observations}

Generate random samples. Assume we have \(n=100\) subjects and \(m=200\)
snps.

\begin{Shaded}
\begin{Highlighting}[]
\NormalTok{n =}\StringTok{ }\DecValTok{100}
\NormalTok{m =}\StringTok{ }\DecValTok{1000}
\NormalTok{z =}\StringTok{ }\KeywordTok{rbinom}\NormalTok{(n,}\DataTypeTok{size =} \DecValTok{2}\NormalTok{, }\DataTypeTok{prob =} \FloatTok{0.5}\NormalTok{)}
\NormalTok{X =}\StringTok{ }\KeywordTok{matrix}\NormalTok{(}\KeywordTok{rep}\NormalTok{(}\DecValTok{0}\NormalTok{,n *}\StringTok{ }\NormalTok{m),}\DataTypeTok{ncol=}\NormalTok{n)}
\NormalTok{snp_afs_0 =}\StringTok{ }\KeywordTok{runif}\NormalTok{(m,}\FloatTok{0.01}\NormalTok{,}\FloatTok{0.5}\NormalTok{)}
\NormalTok{snp_afs_1 =}\StringTok{ }\KeywordTok{runif}\NormalTok{(m,}\FloatTok{0.25}\NormalTok{,}\FloatTok{0.75}\NormalTok{)}
\NormalTok{snp_afs_2 =}\StringTok{ }\KeywordTok{c}\NormalTok{(snp_afs_0[}\DecValTok{1}\NormalTok{:(m/}\DecValTok{2}\NormalTok{)],snp_afs_1[(m/}\DecValTok{2} \NormalTok{+}\StringTok{ }\DecValTok{1}\NormalTok{):m])}
\NormalTok{for(i in }\DecValTok{1}\NormalTok{:n) \{}
  \NormalTok{x_i =}\StringTok{ }\OtherTok{NULL}
  \NormalTok{if(z[i] ==}\StringTok{ }\DecValTok{0}\NormalTok{) \{}
    \NormalTok{x_i =}\StringTok{ }\KeywordTok{rbinom}\NormalTok{(m,}\DataTypeTok{size=}\DecValTok{2}\NormalTok{,}\DataTypeTok{prob=}\NormalTok{snp_afs_0)}
  \NormalTok{\} else if(z[i] ==}\StringTok{ }\DecValTok{1}\NormalTok{) \{}
    \NormalTok{x_i =}\StringTok{ }\KeywordTok{rbinom}\NormalTok{(m,}\DataTypeTok{size=}\DecValTok{2}\NormalTok{,}\DataTypeTok{prob=}\NormalTok{snp_afs_1)}
  \NormalTok{\} else \{}
    \NormalTok{x_i =}\StringTok{ }\KeywordTok{rbinom}\NormalTok{(m,}\DataTypeTok{size=}\DecValTok{2}\NormalTok{,}\DataTypeTok{prob=}\NormalTok{snp_afs_2)}
  \NormalTok{\}}
  \NormalTok{X[,i] =}\StringTok{ }\NormalTok{x_i}
\NormalTok{\}}
\end{Highlighting}
\end{Shaded}

Standardize by subtracting off the row means and dividing by the
standard deviation. Also compute covariance matrix.

\begin{Shaded}
\begin{Highlighting}[]
\NormalTok{X_std =}\StringTok{ }\KeywordTok{t}\NormalTok{(}\KeywordTok{scale}\NormalTok{(}\KeywordTok{t}\NormalTok{(X)))}
\NormalTok{ind_cov =}\StringTok{ }\KeywordTok{cov}\NormalTok{(X_std)}
\NormalTok{ind_cor_x =}\StringTok{ }\KeywordTok{cor}\NormalTok{(X)}

\CommentTok{#approx covariance matrix}
\NormalTok{Xt_X =}\StringTok{ }\KeywordTok{t}\NormalTok{(X_std) %*%}\StringTok{ }\NormalTok{X_std *}\StringTok{ }\DecValTok{1}\NormalTok{/(n}\DecValTok{-1}\NormalTok{)}

\KeywordTok{dim}\NormalTok{(ind_cov)}
\end{Highlighting}
\end{Shaded}

\begin{verbatim}
## [1] 100 100
\end{verbatim}

\begin{Shaded}
\begin{Highlighting}[]
\KeywordTok{dim}\NormalTok{(Xt_X)}
\end{Highlighting}
\end{Shaded}

\begin{verbatim}
## [1] 100 100
\end{verbatim}

\begin{Shaded}
\begin{Highlighting}[]
\KeywordTok{sign}\NormalTok{(Xt_X[}\DecValTok{1}\NormalTok{:}\DecValTok{10}\NormalTok{,}\DecValTok{1}\NormalTok{:}\DecValTok{10}\NormalTok{]) ==}\StringTok{ }\KeywordTok{sign}\NormalTok{(ind_cov[}\DecValTok{1}\NormalTok{:}\DecValTok{10}\NormalTok{,}\DecValTok{1}\NormalTok{:}\DecValTok{10}\NormalTok{])}
\end{Highlighting}
\end{Shaded}

\begin{verbatim}
##        [,1]  [,2] [,3]  [,4] [,5] [,6]  [,7] [,8]  [,9] [,10]
##  [1,]  TRUE  TRUE TRUE  TRUE TRUE TRUE FALSE TRUE  TRUE  TRUE
##  [2,]  TRUE  TRUE TRUE FALSE TRUE TRUE  TRUE TRUE  TRUE  TRUE
##  [3,]  TRUE  TRUE TRUE  TRUE TRUE TRUE  TRUE TRUE  TRUE  TRUE
##  [4,]  TRUE FALSE TRUE  TRUE TRUE TRUE  TRUE TRUE  TRUE FALSE
##  [5,]  TRUE  TRUE TRUE  TRUE TRUE TRUE  TRUE TRUE  TRUE  TRUE
##  [6,]  TRUE  TRUE TRUE  TRUE TRUE TRUE  TRUE TRUE  TRUE  TRUE
##  [7,] FALSE  TRUE TRUE  TRUE TRUE TRUE  TRUE TRUE FALSE  TRUE
##  [8,]  TRUE  TRUE TRUE  TRUE TRUE TRUE  TRUE TRUE  TRUE  TRUE
##  [9,]  TRUE  TRUE TRUE  TRUE TRUE TRUE FALSE TRUE  TRUE  TRUE
## [10,]  TRUE  TRUE TRUE FALSE TRUE TRUE  TRUE TRUE  TRUE  TRUE
\end{verbatim}

\begin{Shaded}
\begin{Highlighting}[]
\NormalTok{Xt_X[}\DecValTok{1}\NormalTok{:}\DecValTok{10}\NormalTok{,}\DecValTok{1}\NormalTok{:}\DecValTok{10}\NormalTok{] /}\StringTok{ }\NormalTok{ind_cov[}\DecValTok{1}\NormalTok{:}\DecValTok{10}\NormalTok{,}\DecValTok{1}\NormalTok{:}\DecValTok{10}\NormalTok{]}
\end{Highlighting}
\end{Shaded}

\begin{verbatim}
##             [,1]      [,2]     [,3]      [,4]     [,5]      [,6]
##  [1,] 12.0682391 22.302392 21.69088 29.520355 28.13376  3.064514
##  [2,] 22.3023915 10.602309 28.02168 -2.653946 18.17600 12.229005
##  [3,] 21.6908815 28.021677 10.73762 31.328131 42.30572 11.904250
##  [4,] 29.5203547 -2.653946 31.32813 10.641677 39.52809 12.815637
##  [5,] 28.1337611 18.175999 42.30572 39.528093 10.94542 11.834780
##  [6,]  3.0645136 12.229005 11.90425 12.815637 11.83478 10.120649
##  [7,] -0.2461114 13.374454 13.15835 13.502245 13.23654 11.177398
##  [8,] 24.7062758 21.994024 34.79571 33.348623 37.75772  5.875213
##  [9,] 24.6949329 33.689256 22.20866 32.926927 24.36417 25.106769
## [10,] 21.8135544 17.811780 28.41561 -8.151546 16.96303 11.981163
##             [,7]      [,8]      [,9]     [,10]
##  [1,] -0.2461114 24.706276 24.694933 21.813554
##  [2,] 13.3744537 21.994024 33.689256 17.811780
##  [3,] 13.1583452 34.795706 22.208662 28.415606
##  [4,] 13.5022451 33.348623 32.926927 -8.151546
##  [5,] 13.2365382 37.757724 24.364171 16.963035
##  [6,] 11.1773980  5.875213 25.106769 11.981163
##  [7,] 10.1506282  4.428662 -1.355092 13.569101
##  [8,]  4.4286620 12.253188 27.224969 19.554674
##  [9,] -1.3550917 27.224969 12.080157 20.987998
## [10,] 13.5691010 19.554674 20.987998 10.645161
\end{verbatim}

Now let us compare the eigenvalues of each

\begin{Shaded}
\begin{Highlighting}[]
\NormalTok{eigen.XT_X =}\StringTok{ }\KeywordTok{eigen}\NormalTok{(Xt_X)}
\NormalTok{eigen.ind_cov =}\StringTok{ }\KeywordTok{eigen}\NormalTok{(ind_cov)}
\NormalTok{eigen.ind_cor =}\StringTok{ }\KeywordTok{eigen}\NormalTok{(}\KeywordTok{cor}\NormalTok{(X_std))}
\NormalTok{eigen.ind_cor_x =}\StringTok{ }\KeywordTok{eigen}\NormalTok{(ind_cor_x)}
\NormalTok{svd.X =}\StringTok{ }\KeywordTok{svd}\NormalTok{(X_std)}

\NormalTok{color =}\StringTok{ }\NormalTok{function(x) \{}
  \NormalTok{a_func =}\StringTok{ }\NormalTok{function(a) \{}
    \NormalTok{if(a ==}\StringTok{ }\DecValTok{0}\NormalTok{) \{}
      \KeywordTok{return}\NormalTok{(}\StringTok{"black"}\NormalTok{)}
    \NormalTok{\} else if(a ==}\StringTok{ }\DecValTok{1}\NormalTok{) \{}
      \KeywordTok{return}\NormalTok{(}\StringTok{"red"}\NormalTok{)}
    \NormalTok{\} else \{}
      \KeywordTok{return}\NormalTok{(}\StringTok{"blue"}\NormalTok{)}
    \NormalTok{\}}
  \NormalTok{\}}
  \KeywordTok{sapply}\NormalTok{(x,}\DataTypeTok{FUN=}\NormalTok{a_func)}
\NormalTok{\}}

\NormalTok{(svd.X$d^}\DecValTok{2} \NormalTok{/}\StringTok{ }\NormalTok{(n -}\StringTok{ }\DecValTok{1}\NormalTok{))[}\DecValTok{1}\NormalTok{:}\DecValTok{10}\NormalTok{]}
\end{Highlighting}
\end{Shaded}

\begin{verbatim}
##  [1] 139.46563  36.75533  14.51172  14.30498  14.09950  13.86599  13.57933
##  [8]  13.46745  13.27360  13.05524
\end{verbatim}

\begin{Shaded}
\begin{Highlighting}[]
\NormalTok{eigen.XT_X$values[}\DecValTok{1}\NormalTok{:}\DecValTok{10}\NormalTok{]}
\end{Highlighting}
\end{Shaded}

\begin{verbatim}
##  [1] 139.46563  36.75533  14.51172  14.30498  14.09950  13.86599  13.57933
##  [8]  13.46745  13.27360  13.05524
\end{verbatim}

\begin{Shaded}
\begin{Highlighting}[]
\KeywordTok{plot}\NormalTok{(svd.X$d^}\DecValTok{2} \NormalTok{/}\StringTok{ }\NormalTok{(n -}\StringTok{ }\DecValTok{1}\NormalTok{),eigen.XT_X$values,}\DataTypeTok{main=}\StringTok{"singluar values^2 / (n-1) = eigenvalues"}\NormalTok{)}
\end{Highlighting}
\end{Shaded}

\includegraphics{PCA_SVD_AND_Mahalanobis_Distance_files/figure-latex/unnamed-chunk-12-1.pdf}

\begin{Shaded}
\begin{Highlighting}[]
\KeywordTok{plot}\NormalTok{(svd.X$v[,}\DecValTok{1}\NormalTok{],svd.X$v[,}\DecValTok{2}\NormalTok{],}\DataTypeTok{col=}\KeywordTok{color}\NormalTok{(z))}
\end{Highlighting}
\end{Shaded}

\includegraphics{PCA_SVD_AND_Mahalanobis_Distance_files/figure-latex/unnamed-chunk-12-2.pdf}

\begin{Shaded}
\begin{Highlighting}[]
\KeywordTok{plot}\NormalTok{(eigen.XT_X$vectors[,}\DecValTok{1}\NormalTok{],eigen.XT_X$vectors[,}\DecValTok{2}\NormalTok{],}\DataTypeTok{col=}\KeywordTok{color}\NormalTok{(z))}
\end{Highlighting}
\end{Shaded}

\includegraphics{PCA_SVD_AND_Mahalanobis_Distance_files/figure-latex/unnamed-chunk-12-3.pdf}

\begin{Shaded}
\begin{Highlighting}[]
\KeywordTok{plot}\NormalTok{(eigen.ind_cov$vectors[,}\DecValTok{1}\NormalTok{],eigen.ind_cov$vectors[,}\DecValTok{2}\NormalTok{],}\DataTypeTok{col=}\KeywordTok{color}\NormalTok{(z))}
\end{Highlighting}
\end{Shaded}

\includegraphics{PCA_SVD_AND_Mahalanobis_Distance_files/figure-latex/unnamed-chunk-12-4.pdf}

\begin{Shaded}
\begin{Highlighting}[]
\KeywordTok{plot}\NormalTok{(eigen.ind_cor$vectors[,}\DecValTok{1}\NormalTok{],eigen.ind_cor$vectors[,}\DecValTok{2}\NormalTok{],}\DataTypeTok{col=}\KeywordTok{color}\NormalTok{(z))}
\end{Highlighting}
\end{Shaded}

\includegraphics{PCA_SVD_AND_Mahalanobis_Distance_files/figure-latex/unnamed-chunk-12-5.pdf}

\begin{Shaded}
\begin{Highlighting}[]
\KeywordTok{plot}\NormalTok{(eigen.ind_cor_x$vectors[,}\DecValTok{1}\NormalTok{],eigen.ind_cor_x$vectors[,}\DecValTok{2}\NormalTok{],}\DataTypeTok{col=}\KeywordTok{color}\NormalTok{(z))}
\end{Highlighting}
\end{Shaded}

\includegraphics{PCA_SVD_AND_Mahalanobis_Distance_files/figure-latex/unnamed-chunk-12-6.pdf}

\begin{Shaded}
\begin{Highlighting}[]
\KeywordTok{plot}\NormalTok{(eigen.ind_cov$vectors[,}\DecValTok{1}\NormalTok{],eigen.XT_X$vectors[,}\DecValTok{1}\NormalTok{])}
\end{Highlighting}
\end{Shaded}

\includegraphics{PCA_SVD_AND_Mahalanobis_Distance_files/figure-latex/unnamed-chunk-12-7.pdf}

\begin{Shaded}
\begin{Highlighting}[]
\KeywordTok{plot}\NormalTok{(eigen.ind_cov$vectors[,}\DecValTok{2}\NormalTok{],eigen.XT_X$vectors[,}\DecValTok{2}\NormalTok{])}
\end{Highlighting}
\end{Shaded}

\includegraphics{PCA_SVD_AND_Mahalanobis_Distance_files/figure-latex/unnamed-chunk-12-8.pdf}

\begin{Shaded}
\begin{Highlighting}[]
\KeywordTok{plot}\NormalTok{(eigen.ind_cov$vectors[,}\DecValTok{3}\NormalTok{],eigen.XT_X$vectors[,}\DecValTok{3}\NormalTok{])}
\end{Highlighting}
\end{Shaded}

\includegraphics{PCA_SVD_AND_Mahalanobis_Distance_files/figure-latex/unnamed-chunk-12-9.pdf}

\begin{Shaded}
\begin{Highlighting}[]
\KeywordTok{plot}\NormalTok{(eigen.ind_cor$vectors[,}\DecValTok{1}\NormalTok{],eigen.XT_X$vectors[,}\DecValTok{1}\NormalTok{])}
\end{Highlighting}
\end{Shaded}

\includegraphics{PCA_SVD_AND_Mahalanobis_Distance_files/figure-latex/unnamed-chunk-12-10.pdf}

\begin{Shaded}
\begin{Highlighting}[]
\KeywordTok{plot}\NormalTok{(eigen.ind_cor$vectors[,}\DecValTok{2}\NormalTok{],eigen.XT_X$vectors[,}\DecValTok{2}\NormalTok{])}
\end{Highlighting}
\end{Shaded}

\includegraphics{PCA_SVD_AND_Mahalanobis_Distance_files/figure-latex/unnamed-chunk-12-11.pdf}

\begin{Shaded}
\begin{Highlighting}[]
\KeywordTok{plot}\NormalTok{(eigen.ind_cor$vectors[,}\DecValTok{3}\NormalTok{],eigen.XT_X$vectors[,}\DecValTok{3}\NormalTok{])}
\end{Highlighting}
\end{Shaded}

\includegraphics{PCA_SVD_AND_Mahalanobis_Distance_files/figure-latex/unnamed-chunk-12-12.pdf}

\begin{Shaded}
\begin{Highlighting}[]
\KeywordTok{plot}\NormalTok{(eigen.ind_cov$values,eigen.XT_X$values)}
\end{Highlighting}
\end{Shaded}

\includegraphics{PCA_SVD_AND_Mahalanobis_Distance_files/figure-latex/unnamed-chunk-12-13.pdf}

\begin{Shaded}
\begin{Highlighting}[]
\KeywordTok{plot}\NormalTok{(eigen.ind_cov$values,eigen.ind_cor$values)}
\end{Highlighting}
\end{Shaded}

\includegraphics{PCA_SVD_AND_Mahalanobis_Distance_files/figure-latex/unnamed-chunk-12-14.pdf}

\begin{Shaded}
\begin{Highlighting}[]
\KeywordTok{plot}\NormalTok{(eigen.ind_cor$values,eigen.ind_cor_x$values)}
\end{Highlighting}
\end{Shaded}

\includegraphics{PCA_SVD_AND_Mahalanobis_Distance_files/figure-latex/unnamed-chunk-12-15.pdf}

\begin{Shaded}
\begin{Highlighting}[]
\KeywordTok{plot}\NormalTok{(eigen.ind_cor_x$vectors[,}\DecValTok{1}\NormalTok{],eigen.XT_X$vectors[,}\DecValTok{1}\NormalTok{])}
\end{Highlighting}
\end{Shaded}

\includegraphics{PCA_SVD_AND_Mahalanobis_Distance_files/figure-latex/unnamed-chunk-12-16.pdf}

\begin{Shaded}
\begin{Highlighting}[]
\KeywordTok{plot}\NormalTok{(eigen.ind_cor_x$vectors[,}\DecValTok{2}\NormalTok{],eigen.XT_X$vectors[,}\DecValTok{2}\NormalTok{])}
\end{Highlighting}
\end{Shaded}

\includegraphics{PCA_SVD_AND_Mahalanobis_Distance_files/figure-latex/unnamed-chunk-12-17.pdf}
The eigenvalues from svd match those from the eigenvalues from the
matrix Xt\_X

Now let us see how

Higher varying genes have higher weights assigned to them

\begin{Shaded}
\begin{Highlighting}[]
\KeywordTok{plot}\NormalTok{((svd.X$u[,}\DecValTok{1}\NormalTok{])^}\DecValTok{2}\NormalTok{,}\KeywordTok{apply}\NormalTok{(X,}\DataTypeTok{MARGIN=}\DecValTok{1}\NormalTok{,}\DataTypeTok{FUN=}\NormalTok{var))}
\end{Highlighting}
\end{Shaded}

\includegraphics{PCA_SVD_AND_Mahalanobis_Distance_files/figure-latex/unnamed-chunk-13-1.pdf}

\begin{Shaded}
\begin{Highlighting}[]
\KeywordTok{cor}\NormalTok{((svd.X$u[,}\DecValTok{1}\NormalTok{])^}\DecValTok{2}\NormalTok{,}\KeywordTok{apply}\NormalTok{(X,}\DataTypeTok{MARGIN=}\DecValTok{1}\NormalTok{,}\DataTypeTok{FUN=}\NormalTok{var))}
\end{Highlighting}
\end{Shaded}

\begin{verbatim}
## [1] 0.5794214
\end{verbatim}

\begin{Shaded}
\begin{Highlighting}[]
\NormalTok{best_snps_for_comp_1 =}\StringTok{ }\KeywordTok{sort}\NormalTok{(svd.X$u[,}\DecValTok{1}\NormalTok{],}\DataTypeTok{decreasing=}\NormalTok{T,}\DataTypeTok{index.return=}\NormalTok{T)$ix}
\NormalTok{best_snps_for_comp_2 =}\StringTok{ }\KeywordTok{sort}\NormalTok{(svd.X$u[,}\DecValTok{2}\NormalTok{],}\DataTypeTok{decreasing=}\NormalTok{T,}\DataTypeTok{index.return=}\NormalTok{T)$ix}

\NormalTok{genes_ranked_by_var =}\StringTok{ }\KeywordTok{sort}\NormalTok{(}\KeywordTok{apply}\NormalTok{(X,}\DataTypeTok{MARGIN=}\DecValTok{1}\NormalTok{,}\DataTypeTok{FUN=}\NormalTok{var),}\DataTypeTok{index.return=}\NormalTok{T,}\DataTypeTok{decreasing=}\NormalTok{T)}
\KeywordTok{ggplot}\NormalTok{(}\KeywordTok{data.frame}\NormalTok{()) +}\StringTok{ }\KeywordTok{geom_jitter}\NormalTok{(}\KeywordTok{aes}\NormalTok{(}\DataTypeTok{x=}\NormalTok{X[best_snps_for_comp_1[}\DecValTok{1}\NormalTok{],],}\DataTypeTok{y=}\NormalTok{X[best_snps_for_comp_2[}\DecValTok{1}\NormalTok{],],}\DataTypeTok{color=}\KeywordTok{color}\NormalTok{(z)),}\DataTypeTok{width =} \FloatTok{0.25}\NormalTok{,}\DataTypeTok{height=}\FloatTok{0.25}\NormalTok{)}
\end{Highlighting}
\end{Shaded}

\includegraphics{PCA_SVD_AND_Mahalanobis_Distance_files/figure-latex/unnamed-chunk-13-2.pdf}

\begin{Shaded}
\begin{Highlighting}[]
\KeywordTok{ggplot}\NormalTok{(}\KeywordTok{data.frame}\NormalTok{()) +}\StringTok{ }\KeywordTok{geom_jitter}\NormalTok{(}\KeywordTok{aes}\NormalTok{(}\DataTypeTok{x=}\NormalTok{X[best_snps_for_comp_2[}\DecValTok{1}\NormalTok{],],}\DataTypeTok{y=}\NormalTok{X[best_snps_for_comp_2[}\DecValTok{2}\NormalTok{],],}\DataTypeTok{color=}\KeywordTok{color}\NormalTok{(z)),}\DataTypeTok{width =} \FloatTok{0.25}\NormalTok{,}\DataTypeTok{height=}\FloatTok{0.25}\NormalTok{)}
\end{Highlighting}
\end{Shaded}

\includegraphics{PCA_SVD_AND_Mahalanobis_Distance_files/figure-latex/unnamed-chunk-13-3.pdf}

\begin{Shaded}
\begin{Highlighting}[]
\KeywordTok{ggplot}\NormalTok{(}\KeywordTok{data.frame}\NormalTok{()) +}\StringTok{ }\KeywordTok{geom_density}\NormalTok{(}\KeywordTok{aes}\NormalTok{(}\DataTypeTok{x=}\NormalTok{X[best_snps_for_comp_1[}\DecValTok{1}\NormalTok{],],}\DataTypeTok{color=}\KeywordTok{color}\NormalTok{(z)))}
\end{Highlighting}
\end{Shaded}

\includegraphics{PCA_SVD_AND_Mahalanobis_Distance_files/figure-latex/unnamed-chunk-13-4.pdf}

\begin{Shaded}
\begin{Highlighting}[]
\KeywordTok{ggplot}\NormalTok{(}\KeywordTok{data.frame}\NormalTok{()) +}\StringTok{ }\KeywordTok{geom_density}\NormalTok{(}\KeywordTok{aes}\NormalTok{(}\DataTypeTok{x=}\NormalTok{X[best_snps_for_comp_1[}\DecValTok{2}\NormalTok{],],}\DataTypeTok{color=}\KeywordTok{color}\NormalTok{(z)))}
\end{Highlighting}
\end{Shaded}

\includegraphics{PCA_SVD_AND_Mahalanobis_Distance_files/figure-latex/unnamed-chunk-13-5.pdf}

\begin{Shaded}
\begin{Highlighting}[]
\KeywordTok{ggplot}\NormalTok{(}\KeywordTok{data.frame}\NormalTok{()) +}\StringTok{ }\KeywordTok{geom_density}\NormalTok{(}\KeywordTok{aes}\NormalTok{(}\DataTypeTok{x=}\NormalTok{X[best_snps_for_comp_2[}\DecValTok{1}\NormalTok{],],}\DataTypeTok{color=}\KeywordTok{color}\NormalTok{(z)))}
\end{Highlighting}
\end{Shaded}

\includegraphics{PCA_SVD_AND_Mahalanobis_Distance_files/figure-latex/unnamed-chunk-13-6.pdf}

\begin{Shaded}
\begin{Highlighting}[]
\KeywordTok{ggplot}\NormalTok{(}\KeywordTok{data.frame}\NormalTok{()) +}\StringTok{ }\KeywordTok{geom_density}\NormalTok{(}\KeywordTok{aes}\NormalTok{(}\DataTypeTok{x=}\NormalTok{X[best_snps_for_comp_2[}\DecValTok{2}\NormalTok{],],}\DataTypeTok{color=}\KeywordTok{color}\NormalTok{(z)))}
\end{Highlighting}
\end{Shaded}

\includegraphics{PCA_SVD_AND_Mahalanobis_Distance_files/figure-latex/unnamed-chunk-13-7.pdf}


\end{document}
